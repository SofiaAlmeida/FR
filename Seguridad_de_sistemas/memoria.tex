%%%
% Plantilla de Memoria
% Modificación de una plantilla de Latex de Nicolas Diaz para adaptarla 
% al castellano y a las necesidades de escribir informática y matemáticas.
%
% Editada por: Mario Román
%
% License:
% CC BY-NC-SA 3.0 (http://creativecommons.org/licenses/by-nc-sa/3.0/)
%%%

%%%%%%%%%%%%%%%%%%%%%%%%%%%%%%%%%%%%%%%%%
% Thin Sectioned Essay
% LaTeX Template
% Version 1.0 (3/8/13)
%
% This template has been downloaded from:
% http://www.LaTeXTemplates.com
%
% Original Author:
% Nicolas Diaz (nsdiaz@uc.cl) with extensive modifications by:
% Vel (vel@latextemplates.com)
%
% License:
% CC BY-NC-SA 3.0 (http://creativecommons.org/licenses/by-nc-sa/3.0/)
%
%%%%%%%%%%%%%%%%%%%%%%%%%%%%%%%%%%%%%%%%%

%----------------------------------------------------------------------------------------
%	PAQUETES Y CONFIGURACIÓN DEL DOCUMENTO
%----------------------------------------------------------------------------------------

%%% Configuración del papel.
% microtype: Tipografía.
% mathpazo: Usa la fuente Palatino.
\documentclass[a4paper, 11pt]{article}
\usepackage[protrusion=false,expansion=false]{microtype}
\usepackage{mathpazo}

% Indentación de párrafos para Palatino
\setlength{\parindent}{0pt}
  \parskip=8pt
\linespread{1.05} % Change line spacing here, Palatino benefits from a slight increase by default


%%% Castellano.
% noquoting: Permite uso de comillas no españolas.
% lcroman: Permite la enumeración con numerales romanos en minúscula.
% fontenc: Usa la fuente completa para que pueda copiarse correctamente del pdf.
\usepackage[spanish,es-noquoting,es-lcroman]{babel}
\usepackage[utf8]{inputenc}
\usepackage[T1]{fontenc}
\selectlanguage{spanish}


%%% Gráficos
\usepackage{graphicx} % Required for including pictures
\usepackage{wrapfig} % Allows in-line images
\usepackage[usenames,dvipsnames]{color} % Coloring code


%%% Matemáticas
\usepackage{amsmath}
\usepackage{amsthm} 

%%% Bibliografía
\makeatletter
\renewcommand\@biblabel[1]{\textbf{#1.}} % Change the square brackets for each bibliography item from '[1]' to '1.'
\renewcommand{\@listI}{\itemsep=0pt} % Reduce the space between items in the itemize and enumerate environments and the bibliography

%----------------------------------------------------------------------------------------
%	ENTORNO EJERCICIOS
%----------------------------------------------------------------------------------------
\newtheoremstyle{}
  {\topsep}   % ABOVESPACE
  {\topsep}   % BELOWSPACE
  {\itshape}  % BODYFONT
  {0pt}       % INDENT (empty value is the same as 0pt)
  {\bfseries} % HEADFONT
  {.}         % HEADPUNCT
  {\newline} % HEADSPACE
  {}          % CUSTOM-HEAD-SPEC


\theoremstyle{plain}
\newtheorem{nej}{Ejercicio}[section]

%----------------------------------------------------------------------------------------
%	TÍTULO
%----------------------------------------------------------------------------------------
% Configuraciones para el título.
% El título no debe editarse aquí.
\renewcommand{\maketitle}{
  \begin{flushright} % Right align
  
  {\LARGE\@title} % Increase the font size of the title
  
  \vspace{50pt} % Some vertical space between the title and author name
  
  {\large\@author} % Author name
  \\\@date % Date
  \vspace{40pt} % Some vertical space between the author block and abstract
  \end{flushright}
}

%% Título
\title{\textbf{Software para test de seguridad}\\ % Title
Fundamentos de Redes} % Subtitle

\author{\textsc{Sofía Almeida Bruno\\Fernando de la Hoz Moreno} % Author
\\{\textit{Universidad de Granada}}} % Institution

\date{\today} % Date



%----------------------------------------------------------------------------------------
%	DOCUMENTO
%----------------------------------------------------------------------------------------

\begin{document}

\maketitle % Print the title section
\newpage
\tableofcontents
\newpage

%%%%%%%%%%%%%%%%%%%%%%%%%%%%%%%%%%%%%%%%%%%%%%%%%%%%%%%%%%%%%%%%%%%%%%%%%%%%%%%
%%%%%                           INTRODUCCIÓN                              %%%%%
%%%%%%%%%%%%%%%%%%%%%%%%%%%%%%%%%%%%%%%%%%%%%%%%%%%%%%%%%%%%%%%%%%%%%%%%%%%%%%%
\section{Introducción}

Los test de seguridad son un conjunto de metodologías y técnicas que permiten simular ataques maliciosos para evaluar el nivel de seguridad de una red, sistema o software y así poder prevenirlos.\\

Podemos distinguir dos tipos de intrusión según desde donde se realice el acceso:
\begin{itemize}
	\item \textbf{Caja negra}: el auditor solo dispone de información pública, realiza el test "desde fuera de la red", simulando un ataque con los recursos que tendría una persona externa.
		\item \textbf{Caja blanca}: el auditor hace el test con información sobre el sistema, desde dentro de la red, con diferentes privilegios.
\end{itemize}

A continuación, explicaremos distintos tipos de herramientas utilizadas para evaluar diferentes aspectos de la seguridad, una no sustituye a otra, son complementarias.


%%%%%%%%%%%%%%%%%%%%%%%%%%%%%%%%%%%%%%%%%%%%%%%%%%%%%%%%%%%%%%%%%%%%%%%%%%%%%%%
%%%%%                        TIPOS DE HERRAMIENTAS                        %%%%%
%%%%%%%%%%%%%%%%%%%%%%%%%%%%%%%%%%%%%%%%%%%%%%%%%%%%%%%%%%%%%%%%%%%%%%%%%%%%%%%
\section{Tipos de herramientas}

Clasificaremos las diferentes utilidades según el objetivo que persiguen.

\subsection{Recopilar información}
Estos programas permiten capturar toda la información posible sobre el objetivo, direcciones IP, posibles nombres de usuario, DNS, $\dots$ Se puede obtener información de forma activa, enviando tráfico hacia la red objetivo o escaneando puertos TCP/UDP, y de forma pasiva, buscando en los recursos públicos con este objetivo y disminuyendo la probabilidad de ser detectados.

\subsection{Análisis de vulnerabilidades}

Estas herramientas permiten conocer las máquinas que están disponibles de la red, tipo y versión del sistema operativo, puertos abiertos, servicios en funcionamiento.

\subsubsection{Nmap}
Un ejemplo de este tipo de herramientas es \textit{nmap} (network mapper)

% Completar con información del pdf específico

\subsection{Aplicaciones Web}
\subsection{Evaluación de Bases de Datos}
\subsection{Ataques de contraseñas}
\subsection{Ataques Wireless}
\subsection{Ingeniería inversa}
%REVIEW: cambiar título
\subsection{Herramientas de explotación}
\subsection{Sniffing}
\subsection{Mantener el acceso}


\section{Bibliografía}

\end{document}