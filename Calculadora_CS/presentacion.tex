\documentclass[spanish]{beamer}

%%% CODIFICACIÓN

\usepackage[utf8]{inputenc}
\usepackage[spanish]{babel}
\usepackage{graphics,tikz}

%%% FUENTES

\usepackage[T1]{fontenc}
\usepackage{newtxsf} % Fuente de matemáticas

\setbeamertemplate{navigation symbols}{}

%%% COLORES

\definecolor{background}{RGB}{237,237,237}
\definecolor{text}{RGB}{78,78,78}
\definecolor{accent}{RGB}{255, 153, 153}
\definecolor{accent}{RGB}{255, 102, 102}

\setbeamerfont{framesubtitle}{size=\normalfont\tiny}
\setbeamercolor{framesubtitle}{fg=white}


%%% AJUSTES DE BEAMER

% ¿Negrita en el título de diapositiva o no?
%\setbeamertemplate{frametitle}{\color{accent}\vspace*{1cm}\bfseries\insertframetitle\par\vskip-6pt}

\setbeamertemplate{frametitle}{\color{accent}\vspace*{1cm}\insertframetitle\par\vskip-6pt}

\setbeamertemplate{itemize items}[circle] % Viñetas de itemize

%%% CONFIGURACIÓN DE COLORES DE BEAMER

\setbeamercolor{background canvas}{bg=background}
\setbeamercolor{normal text}{fg=text}
\setbeamercolor{alerted text}{fg=accent}
\setbeamercolor{block title}{fg=accent}
\setbeamercolor{alerted text}{fg=accent}
\setbeamercolor{itemize item}{fg=accent}
\setbeamercolor{enumerate item}{fg=accent}
\setbeamercolor*{title}{fg=accent}
\setbeamercolor{qed symbol}{fg=accent}
\usebeamercolor[fg]{normal text}

%%% PGFPLOTSTABLE

\usepackage{pgfplotstable}


\pgfplotstableset{
columns/0/.style={
     column name={Elementos},
   },
columns/1/.style={
     column name={Tiempo en segundos},
   },
}

%%% INFORMACIÓN DEL DOCUMENTO

\title{Fundamentos de Redes}
\subtitle{Definición e implementación de un protocolo de aplicación}
\author{Sofía Almeida Bruno\\ Fernando de la Hoz Moreno}

\begin{document}

\maketitle
\begin{frame}{Descripción de la aplicación - Calculadora}
	\begin{itemize}
		\item Clientes: usuarios que solicitan realizar un cálculo
		\item Servidor: calculadora que atiende las peticiones 
		\item Sockets TCP
\end{itemize}
	
	 

\end{frame}

\begin{frame}{Diagrama de estados del servidor}
	\begin{figure}[h]
		\centering
		\includegraphics[scale=0.55]{./diagrama}
	\end{figure}
\end{frame}


\begin{frame}{Mensajes que intervienen}
	\begin{table}[]
\centering
\label{my-label}
\begin{tabular}{lll}
\hline
\multicolumn{1}{|l|}{Código} & \multicolumn{1}{l|}{Cuerpo} & \multicolumn{1}{l|}{Descripción}   \\ \hline
001                          & HOLA                        & Confirmación de conexión realizada \\
002                          & OPC                         & Opción elegida                     \\
003                          & OK                          & Elección verificada                \\
004                          & ERR                         & Opción incorrecta                  \\
005                          & DAT                         & Datos con los que operar           \\
006                          & RES                         & Resultado de la operación          \\
007                          & SI                          & Realizar otra operación            \\
008                          & NO                          & No realizar otra operación         \\
009                          & ADIOS                       & Finalización de la conexión       
\end{tabular}
\end{table}
\end{frame}

\end{document}